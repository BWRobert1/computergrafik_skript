\documentclass[a4paper,12pt]{scrartcl}

\usepackage[latin1]{inputenc}
\usepackage{amsfonts}
\usepackage{amsmath}
\usepackage{amssymb}
\usepackage{amsthm}
\usepackage{color}
\usepackage[ngerman]{babel}
\usepackage[pdftex]{graphicx}
%\usepackage[T1]{fontenc}

\pagestyle{empty}

%\topmargin20mm
\oddsidemargin0mm
\parindent0mm
\parskip2mm
\textheight25.5cm
\textwidth15.8cm
\unitlength1mm

\newcommand{\E}{\mathbb{E}}
\newcommand{\Hy}{\mathbb{H}}
\newcommand{\N}{\mathbb{N}}
\newcommand{\RR}{\mathbb{R}}
\newcommand{\Z}{\mathbb{Z}}
\newcommand{\Q}{\mathbb{Q}}
\newcommand{\C}{\mathbb{C}}
\newcommand{\K}{\mathbb{K}}
\newcommand*\ee{\mathrm{e}}
\newcommand*\ii{\mathrm{i}}
\newcommand*\re{\mathrm{Re}}
\newcommand*\im{\mathrm{Im}}
\newcommand*\id{\mathrm{id}}
\newcommand*\glnr{\mathrm{\it GL}(n,\R)}
\newcommand*\slnr{\mathrm{\it SL}(n,\R)}
\newcommand*\on{\mathrm{\it O}(n)}
\newcommand*\son{\mathrm{\it SO}(n)}
\newcommand*\rang{\mathrm{Rang}}
\newcommand*\grad{\mathrm{grad~}}
\newcommand*\dive{\mathrm{div~}}
\newcommand*\sym{\mathrm{Sym}}
\newcommand*\spur{\mathrm{Spur}}
\newcommand*\isom{\mathrm{Isom}}
\newcommand*\bmo{{\bf o}}
\newcommand*\bmu{{\bf u}}
\newcommand*\bmw{{\bf w}}
\newcommand*\bmx{{\bf x}}
\newcommand*\bmb{{\bf b}}
\newcommand*\bmc{{\bf c}}
\newcommand*\bmy{{\bf y}}
\newcommand*\bma{{\bf a}}
\newcommand*\bmp{{\bf p}}
\newcommand*\bmm{{\bf m}}
\newcommand{\cls}{\color{blue}}


%\author{Gabriele Link}

\begin{document}


%\vspace*{-20mm}
% \vspace*{-35mm}  %FUER pdflatex

%\begin{picture}(4,2)
% Die folgenden 2 Zeilen sind als Kommentar zu 
% kennzeichnen, falls kein Logo gewuenscht
%\put(0,0){
%\includegraphics[width=4cm]{kit_logo_de_farbig.jpg}}
%\end{picture}\hfill

%\hspace{9cm}


\section*{\large Einf\"uhrung in die Computergrafik \\\vspace*{5mm}
                  \normalsize  Aufgabenblatt 1 }
\hrule
\hrule
\vspace{4mm}
%\includegraphics[width=0.8\textwidth]{sampleplot.pdf}



{\bf Aufgabe 1. Matrizen. \hfill (4 Punkte)}

Zeigen Sie, dass f\"ur alle Matrizen $A  \in M^{n \times m}$ ,  alle Vektoren $v,w \in \mathbb{R}^m$ und alle $\lambda \in \mathbb{R}$ gilt:
\begin{itemize}
\item[(a)]   $A(v +w) =  Av +Aw $.
\item[(b)] $A (\lambda v) = \lambda A v$. 
\end{itemize}

 
\vspace*{8mm}


{\bf Aufgabe 2.  Basisdarstellung und Basiswechsel.\hfill (4 Punkte)}

Gegeben seien die Basen 
\begin{align*}
B_1 := \Biggl \{  
\begin{pmatrix} 0 \\  0 \\ 1 \end{pmatrix},
\begin{pmatrix} 1\\  0 \\  0  \end{pmatrix},
\begin{pmatrix} 0 \\ 1 \\  0  \end{pmatrix}
  \Biggr \}
, \;
B_2 := \Biggl \{  
 \frac{1}{\sqrt{2}} \cdot \begin{pmatrix} 1 \\  0 \\ 1 \end{pmatrix},
\begin{pmatrix} 0\\  1 \\  0  \end{pmatrix},
\frac{1}{\sqrt{2}} \cdot \begin{pmatrix} -1 \\ 0  \\  1  \end{pmatrix}
  \Biggr \} 
\end{align*}
des $\mathbb{R}^3$ und $v =\begin {pmatrix} 1 \\  0 \\ 0 \end{pmatrix}$ .
\begin{itemize}\itemsep0pt
\item[(a)] Zeigen Sie, dass $B_2$ eine Orthonormalbasis ist.
\item[(b)]  Berechnen Sie  $\theta_{B_1} (v)$ und  $\theta_{B_2}(v)$.
\item[(c)] Bestimmen Sie die Basiswechselmatrix $M_{B_1}^{B_2}$.
\end{itemize}


 
\vspace*{4mm}

{\bf Aufgabe 3. Affine Basis und affiner Basiswechsel. \hfill (4 Punkte)}

Gegeben seien die affine Basen $(p_1, B_1)$ und $(p_2, B_2)$ des $\mathbb{A}^3$ mit
$B_1$, $B_2$ aus Aufgabe 2, $p_1 = \begin{pmatrix} 0 \\  1 \\ 0 \end{pmatrix}$,  $p_2 = \begin{pmatrix} 0 \\  0 \\ 1 \end{pmatrix}$ und $v= \begin{pmatrix} 1 \\  0 \\ 0 \end{pmatrix}$. 
\begin{itemize}\itemsep0pt
\item[(a)] Berechnen Sie $\theta_{(p_1, B_1)} (v)$ und  $\theta_{(p_2, B_2)}(v)$.
\item[(b)] Es sei $\theta_{(p_1, B_1)} (w) = \begin{pmatrix} 0 \\  0 \\ 1 \end{pmatrix}$. Berechnen Sie   $\theta_{(p_2, B_2)} (w)$.
\end{itemize}



\vspace*{4mm}


{\bf Aufgabe 4. Kreuzprodukt. \hfill (4 Punkte)}

Gegeben seien die Vektoren 
$x=\begin{pmatrix} x_1\\ x_2\\ x_3 \end{pmatrix}$, $y=\begin{pmatrix} y_1\\ y_2\\ y_3 \end{pmatrix}$,  $z=\begin{pmatrix} z_1\\ z_2\\ z_3 \end{pmatrix} \in\RR^3$. Zeigen Sie:
\begin{itemize}
\item [(a)] $x  \times y = -y\times x$.
\item[(b)]
$x$ und $y$ sind genau dann linear abh"angig, wenn $x\times y=0$ ist. 
\item[(a)] $\langle x, x \times y \rangle = \langle y, x\times y \rangle = 0$.  
\end{itemize}

\vspace*{4mm}

\vfill

\hrule


\end{document}
